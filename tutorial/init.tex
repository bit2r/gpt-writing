
%%==============================================================================
%% load packages
%%==============================================================================
\usepackage[utf8]{inputenc}
\usepackage{setspace}
\usepackage{tocloft}
\usepackage{makeidx}                        % 찾아보기 (색인) 정의를 위해
\usepackage{parskip}
\usepackage[hangul]{xetexko}
% \usepackage{listings}                       % shell script code출력을 위함
% \usepackage[framemethod=tikz]{mdframed}
% \usepackage[unicode]{hyperref}
% \usepackage{multirow}
% \usepackage[many]{tcolorbox}
% \usepackage{makecell}
% \usepackage{environ}
% \usepackage[tikz]{bclogo}
% \usepackage{tikz}
% \usepackage{lastpage}
% \usepackage{fontawesome5}


%%==============================================================================
%% 폰트 정의
%%==============================================================================
%% 라틴 셰리프
% https://github.com/stipub/stixfonts
\setmainfont[ExternalLocation=fonts/STIXTwoText/]{STIXTwoText-Regular.otf}[%
  Ligatures=TeX,
  BoldFont=STIXTwoText-Bold.otf,
  ItalicFont=STIXTwoText-Italic.otf,
  BoldItalicFont=STIXTwoText-BoldItalic.otf
]

%% 라틴 산셰리프
% https://www.1001fonts.com/nimbus-sans-l-font.html
\setsansfont[ExternalLocation=fonts/Nimbus Sans L/]{NimbusSanL-Reg.otf}[%
  Ligatures=TeX,
  BoldFont=NimbusSanL-Bol.otf,
  ItalicFont=NimbusSanL-RegIta.otf,
  BoldItalicFont=NimbusSanL-BolIta.otf
]

%% 한국어 셰리프
\setmainhangulfont[ExternalLocation=fonts/KOPUBWORLD_OTF_FONTS/]{KoPubWorld Batang_Pro Light.otf}[%
  Ligatures=TeX,
  BoldFont=KoPubWorld Batang_Pro Bold.otf,
  ItalicFont=KoPubWorld Batang_Pro Light.otf,
  ItalicFeatures = {FakeSlant = 0.167},
  BoldItalicFont=KoPubWorld Batang_Pro Bold.otf,
  ItalicFeatures = {FakeSlant = 0.167}
]

%% 한국어 산셰리프
\setsanshangulfont[ExternalLocation=fonts/KOPUBWORLD_OTF_FONTS/]{KoPubWorld Dotum_Pro Light.otf}[%
  Ligatures=TeX,
  BoldFont=KoPubWorld Dotum_Pro Bold.otf,
  ItalicFont=KoPubWorld Dotum_Pro Light.otf,
  ItalicFeatures = {FakeSlant = 0.167},
  BoldItalicFont=KoPubWorld Dotum_Pro Bold.otf,
  ItalicFeatures = {FakeSlant = 0.167}
]

%% 한자
\setmainhanjafont[ExternalLocation=fonts/KOPUBWORLD_OTF_FONTS/]{KoPubWorld Dotum_Pro Light.otf}[%
  Ligatures=TeX,
  BoldFont=KoPubWorld Dotum_Pro Bold.otf,
  ItalicFont=KoPubWorld Dotum_Pro Light.otf,
  BoldItalicFont=KoPubWorld Dotum_Pro Bold.otf
]

%% 모노스페이스
\setmonofont[ExternalLocation=fonts/D2Coding/]{D2Coding-Ver1.3.2-20180524.ttf}[%
  Scale=0.95,
  Ligatures=TeX,
  BoldFont=D2CodingBold-Ver1.3.2-20180524.ttf,
  ItalicFont=D2Coding-Ver1.3.2-20180524.ttf,
  ItalicFeatures = {FakeSlant = 0.167},
  BoldItalicFont=D2CodingBold-Ver1.3.2-20180524.ttf,
  BoldItalicFeatures = {FakeSlant = 0.167}
]

%% 수식
\setmathfont[ExternalLocation=fonts/STIXTwoText/]{STIXTwoMath-Regular.otf}


%% 기호글꼴 명령 - 라틴 문자나 CJK 기호를 어떤 폰트로 식자할 것인가.
\xetexkofontregime{latin}%
  [alphs=latin, puncts=latin, colons=latin, parens=latin, cjksymbols=hangul]
\xetexkofontregime{hangul}%
  [alphs=latin, puncts=latin, colons=latin, parens=latin, cjksymbols=hangul]


%%==============================================================================
%% 장평/자간/줄간격 등
%%==============================================================================
%% 줄간격 정의
\linespread{1.5}


%%==============================================================================
%% 절(section)과 서브절(subsection) 타이틀을 돋움체(sans-serif)로 바꾸기
%%==============================================================================
%% Rmarkdown과 titlesec 패키지가 호환되지 않는 이슈가 있음.
%% 아래 두줄의 명령을 입력하지 않으면 에러가 발생함
%% 문제의 원인:
%% https://stackoverflow.com/questions/40439701/cant-knit-to-pdf-with-custom-styles
%% 문제의 해결
%% https://github.com/rstudio/bookdown/issues/677
\let\paragraph\oldparagraph
\let\subparagraph\oldsubparagraph

\usepackage{titlesec}
\titleformat{\section}
  {\sffamily\selectfont\Large\bfseries}{\thesection}{1em}{}
\titleformat{\subsection}
  {\sffamily\selectfont\large\bfseries}{\thesubsection}{1em}{}



%%------------------------------------------------------------------------------
%------ 차례 작성
%%------------------------------------------------------------------------------
%% \makeindex

